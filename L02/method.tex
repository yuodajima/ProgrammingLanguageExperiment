%実験方法
実験は以下の方法で行った.
\begin{enumerate}
\item 課題3
  \begin{enumerate}
  \item 中置記法の式の文字列を用意し,式中のトークンについて優先順位を数値でつけた.
  \item 各トークンをスタックを通じて優先順位の高いものから取り出す.
  \item 出力結果が逆ポーランド記法になることを確認した.
  \end{enumerate}
\item 課題4
  \begin{enumerate}
  \item ポインタで二分木を作成し,数値データを格納する.
  \item 格納は絶対値が小さい値を左の子に,大きい値を右の子に格納する.
  \item 格納した数値を絶対値昇順ソートで並んだ順で和を取る.
  \end{enumerate}
\end{enumerate}
データセットは講義サイトに用意してあったものを使用した.また,トークンと優先順位の対応は以下の通りとした.

\begin{table}[htb]
  \begin{center}
    \caption{トークンと優先順位}
    \begin{tabular}{|c||c|} \hline
      トークン & 優先順位 \\\hline
      非演算子 & 5        \\
      \verb|(|& 4        \\
      \verb|*|,/     & 3 \\
      +,-     & 2        \\
      \verb|)|& 1        \\
      =       & 0        \\\hline 
    \end{tabular}
  \end{center}
\end{table}



