結果について,以下のことが見受けられた.
\begin{enumerate}
\item コンテナが配列であるか線形リストであるかは実行結果の差に寄与しない.
\item 絶対値ソートの有無は実行結果の差に寄与する.
\end{enumerate}

ここで.絶対値ソートの有るデータセットと無いデータセットで和を取る様子を比較し,結果に差が生まれる原因を考察する.
なお,result1-1.txtとresult1-2.txtとでソートの有無による和の様子に変化がないため,result1-1.texのみで考える.

90行目のソート無しデータの和[3番目まで]を見ると,ここで10000000000000000.000000に23.000000が加えられた10000000000000023.000000が本来表示されるはずであるが,実際には24.000000が加わった10000000000000024.000000となっている.

また,95行目のソート無しデータの和[4番目まで]を見ると,10000000000000024.000000から6.400000が引かれた10000000000000017.600000になるところが10000000000000018.000000となっており.計算に誤差が生まれている.

いずれも左から17桁目で数値の切り上げによる丸めが行われていることから,C言語におけるdouble型の有効数字は16桁であることが考えられる.

絶対値昇順ソートが行われたデータセットの和は,絶対値が近い数値同士で計算することにより16桁を超えるような桁数の急激な増大が避けられ,更に昇順により巨大な数値の計算の後に有効桁数を超えた小さな値の計算が行われることを未然に防ぐことが可能になるため,情報落ちを起こす可能性が低くなると考えられる.

これ以降にも計算の誤差が存在する場面が存在し,その誤差の集積の結果として最後のデータまで足しきった際に2.000000の差がソート無しのデータセットと有りのデータセットで生まれている.ソートなしの結果のほうが値が大きいのは,丸めが切り上げによって行われているからである.







