\section{課題1}

\subsection{その1}
数値型のデータが並んだ木を探索し,特定の数値と一致するものがあれば\#tを,なければ\#fを返す関数tree-member?関数を作成.
treeを木,valueを探す値としてtree-member(tree, value)のアルゴリズムは次の通り.

\begin{enumerate}
\item treeが空であれば\#fを返す.
\item treeがペア対であれば,treeの先頭と後続に対してtree-member?を適用しorをとる.
\item treeがそれ以外(つまり数値)ならばtreeとvalueの一致を確認する.
\end{enumerate}

2でtree-member?の返り値にorを取ることで,一致する値を異なる階層の中で一つでも見つければ必ず\#tを返す.
\#fは一つも見つからなかった場合のみに返り,orは値の探索に都合が良い.
そして2でのtree-member?の再帰が木の中の走査の役割を持っている.
以下実行例.

\begin{quote}

  (define TREE '(1 (2 (3 4)) 6 (7 8 9)))
  
  (define MYTREE '((3 5) 3 7 (4 5 7) 8))
  
  
  (tree-member? TREE 4)
  
  (tree-member? TREE 12)
  
  (tree-member? MYTREE 8)
  
  (tree-member? MYTREE 9)
  
  その1実行結果
  
  ;\#t
  
  ;\#f
  
  ;\#t
  
  ;\#f
  
\end{quote}


\subsection{その2}
木に対してmapを行うmap-tree関数の作成.
functionを関数,treeを木としてmap-tree(function, tree)のアルゴリズムは次の通り.

\begin{enumerate}
\item treeが空であれば()を返す.
\item treeがペア対であれば,treeの先頭と後続に対してmap-treeを適用しconsで連結する.
\item treeがそれ以外(つまり数値)ならばtreeにfunctionを適用する.
\end{enumerate}

大部分はその1と同じアルゴリズムである.
再帰の部分をconsで連結することで元の木と同じ構造をもう一度作って返している.
以下実行例.

\begin{quote}

  (define TREE '(1 (2 (3 4)) 6 (7 8 9)))
  
  (define MYTREE '((3 5) 3 7 (4 5 7) 8))
  
  (map-tree even? TREE)
  
  (map-tree even? MYTREE)
  
  ;その2実行結果
  
  ;(\#f (\#t (\#f \#t)) \#t (\#f \#t \#f))
  
  ;((\#f \#f) \#f \#f (\#t \#f \#f) \#t)
  
\end{quote}

\subsection{その3}
cons,car,cdrを用いないmap-treeであるmap-tree2の作成.
map-tree2(function, tree)のアルゴリズムは次の通り.

\begin{enumerate}
\item treeが空であれば()を返す.
\item treeがペア対であれば,treeに対してmap-tree2を適用するという無名関数をmapする.
\item treeがそれ以外(つまり数値)ならばtreeにfunctionを適用する.
\end{enumerate}

map-tree2を使う無名関数を作成するのはmapの引数に与えたいためである.
つまり1変数の関数を作りたいためである.
mapより,返る木は引数の木と同じ構造になる.
これにより木構造に対するmapが実現できる.
以下実行例.

\begin{quote}

  (define TREE '(1 (2 (3 4)) 6 (7 8 9)))
  
  (define MYTREE '((3 5) 3 7 (4 5 7) 8))

  (map-tree2 odd? TREE)
  
  (map-tree2 odd? MYTREE)
  
  ;その3実行結果
  
  ;(\#t (\#f (\#t \#f)) \#f (\#t \#f \#t))
  
  ;((\#t \#t) \#t \#t (\#f \#t \#t) \#f)

\end{quote}

